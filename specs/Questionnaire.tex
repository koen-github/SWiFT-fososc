%
%

% Last run through spell checker:

% ====================================================================

\documentclass{article}
\usepackage{pifont}
\usepackage{txfonts}
\usepackage{graphicx}
%\graphicspath{{...}}
%\usepackage{makeidx}  % allows for indexgeneration
\usepackage{graphics}
\usepackage{hyperref}
\usepackage{alltt} % added to be able to add commands in a verbatim environment, e.g make italics within a piece of SPARQL

% correct bad hyphenation here
\hyphenation{op-tical net-works semi-conduc-tor}

\begin{document}

\title{Questionnaire: SWiFT Questionnaire System Specification}

\maketitle

\section{Terminology}

See terminology.ods: todo place online somewhere.

\section{Questionnaire}

The questionnaire is to be used for the following things:

\begin{itemize}
   \item [TraitsPlayer] Establishing the traits of the players (age, educational background, sex, etc.)
   \item [ReadNonConstiCheck] Checking if a person has read parts of the game (such as the help/explanation) of the game well. This point doesn't include the consti-text.
   \item ON HOLD As an OCBKC-challenge that can be constructed by the consti-game player.
   \item ON HOLD As an OCBKC-challenge that is generated by a piece of software (e.g.: a similar question, but with different names for constants - you can generate infinitely of these.).
   \item ON HOLD [ConstiPart] As a part of a constitution (e.g.~to increase the probability people actually READ it. The experience learns that many users are not very careful at reading texts. This use case requires the consti game player to be able to construct questionnaires him- or herself.
\end{itemize}

[ReadNonConstiCheck] and [ConstiPart] is because the experience teaches that many users are not very careful at reading texts.

The most general properties that the questionnaire system should have based on the above usea are the following.

% <& &y2013.09.03.21:01:10& I wonder whether I still need [ConstiPart], given the idea to make constitution so small that they can be studied within 10 minutes or so. Then the challenge itself is almost the test that the text has or hasn't been read!>
% <&y2013.09.03.21:03:20& I think at this point that OCBKC challenges are the most important part, but these have not to be generated yet (in the coming increments) by the consti game players themselves>

\begin{itemize}
   \item Storing the answers of the user at a particular moment he filled out the questionnaire.
   \item A questionnaire creator and editor for consti game players.
   \item 
\end{itemize}

\subsection{Session}

A questionnaire session is defined as a user filling out a particular questionnaire during a particular `session': this is a finite or infinite period in time, which may or may not be allowed to be ended by a user by pressing a ``submit'' button.

\subsection{Types of questionnaires}

Non-terminating questionnaires. To these questionnaires the notion ``submit'' doesn't apply. In a sense they exist in one single infinitely long session. Examples are questionnaires about traits of a person: date of birth, sex, etc. People may ``revisit'' the questionnaire to change their answers to the questionnaire. If they make changes, the original answers are overwritten in the database (possibly after they have been put in an archive).

Terminating questionnaires. These questionnaires will be closed after they are ``submitted'', which will end the session. If it is allowed to access the questionnaire again, the answers will be recorded in a different session, and the results stored separate from the answers of previous sessions.

\subsection{Persistence during sessions}

While a questionnaire is being filled out, the results are immediately stored in the underlying database. This is to prevent data loss, if someone's browser session would crash for example. Also, when changes are made to particular questions, these overwrite the original answers in the underlying database (possibly after they have been put in an archive).

\end{document}
